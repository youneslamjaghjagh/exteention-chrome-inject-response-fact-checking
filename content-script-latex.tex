\documentclass[12pt,a4paper]{article}
\usepackage[utf8]{inputenc}
\usepackage[T1]{fontenc}
\usepackage[french]{babel}
\usepackage{geometry}
\usepackage{listings}
\usepackage{xcolor}
\usepackage{courier}
\usepackage{fancyhdr}
\usepackage{titlesec}

\geometry{margin=2cm}

% Configuration des couleurs pour le code JavaScript
\definecolor{jsbackground}{HTML}{1E1E1E}
\definecolor{jstext}{HTML}{D4D4D4}
\definecolor{jskeyword}{HTML}{569CD6}
\definecolor{jsstring}{HTML}{CE9178}
\definecolor{jscomment}{HTML}{6A9955}
\definecolor{jsnumber}{HTML}{B5CEA8}
\definecolor{jsfunction}{HTML}{DCDCAA}
\definecolor{jsvariable}{HTML}{9CDCFE}

% Configuration du style de code
\lstdefinestyle{vscode-dark}{
    backgroundcolor=\color{jsbackground},
    commentstyle=\color{jscomment},
    keywordstyle=\color{jskeyword}\bfseries,
    numberstyle=\tiny\color{gray},
    stringstyle=\color{jsstring},
    basicstyle=\ttfamily\footnotesize\color{jstext},
    breakatwhitespace=false,
    breaklines=true,
    captionpos=b,
    keepspaces=true,
    numbers=left,
    numbersep=5pt,
    showspaces=false,
    showstringspaces=false,
    showtabs=false,
    tabsize=2,
    frame=single,
    frameround=tttt,
    rulecolor=\color{gray},
    xleftmargin=15pt,
    xrightmargin=5pt,
}

% Mots-clés JavaScript
\lstdefinelanguage{JavaScript}{
    keywords={break, case, catch, continue, debugger, default, delete, do, else, false, finally, for, function, if, in, instanceof, new, null, return, switch, this, throw, true, try, typeof, var, void, while, with, const, let, async, await, class, extends, import, export},
    morestring=[b]",
    morestring=[b]',
    morestring=[b]`,
    morecomment=[l]{//},
    morecomment=[s]{/*}{*/},
    sensitive=true
}

\lstset{
    language=JavaScript,
    style=vscode-dark
}

% Configuration des titres
\titleformat{\section}{\Large\bfseries}{\thesection}{1em}{}
\titleformat{\subsection}{\large\bfseries}{\thesubsection}{1em}{}

\title{\textbf{Code JavaScript - Extension d'Analyse de Posts}}
\author{Extension Chrome}
\date{\today}

\begin{document}

\maketitle
\tableofcontents
\newpage

\section{Variables Globales et Configuration}

\begin{lstlisting}
let postIndex = 0;
const pendingPosts = new Set();
const processedPosts = new Set();
const observedPosts = new Set();

const analysisCache = new Map();
const CACHE_KEY = "post_analysis_cache";
const CACHE_EXPIRY = 24 * 60 * 60 * 1000;
\end{lstlisting}

\section{Fonctions de Gestion du Cache}

\subsection{Chargement du Cache}

\begin{lstlisting}
function loadCache() {
  try {
    const cached = localStorage.getItem(CACHE_KEY);
    if (cached) {
      const parsedCache = JSON.parse(cached);
      for (const [key, value] of Object.entries(parsedCache)) {
        if (Date.now() - value.timestamp < CACHE_EXPIRY) {
          analysisCache.set(key, value);
        }
      }
      console.log(`Cache charge : ${analysisCache.size} entrees`);
    }
  } catch (error) {
    console.error("Erreur lors du chargement du cache :", error);
  }
}
\end{lstlisting}

\subsection{Sauvegarde du Cache}

\begin{lstlisting}
function saveCache() {
  try {
    const cacheObject = Object.fromEntries(analysisCache);
    localStorage.setItem(CACHE_KEY, JSON.stringify(cacheObject));
  } catch (error) {
    console.error("Erreur lors de la sauvegarde du cache :", error);
  }
}
\end{lstlisting}

\subsection{Génération de Clé de Cache}

\begin{lstlisting}
function generateCacheKey(text) {
  const normalizedText = text.trim().toLowerCase().replace(/\s+/g, " ");
  return btoa(encodeURIComponent(normalizedText)).substring(0, 32);
}
\end{lstlisting}

\section{Fonctions d'Extraction de Contenu}

\subsection{Extraction de Texte HTML}

\begin{lstlisting}
function extractTextFromHTML(htmlString) {
  const parser = new DOMParser();
  const doc = parser.parseFromString(htmlString, "text/html");
  return doc.body.textContent || "";
}
\end{lstlisting}

\subsection{Extraction du Contenu Principal}

\begin{lstlisting}
function extractMainPostContent(postElement) {
  try {
    const clonedPost = postElement.cloneNode(true);

    const commentSelectors = [
      ".UFIContainer",
      '[data-testid="ufi_comment_list"]',
      ".comments",
      ".UFIRow",
      '.userContent + div[role="button"]',
      ".UFICommentContainer",
      '[aria-label*="comment"]',
      ".comment",
      '[data-testid*="comment"]',
      ".UFILikeLink",
      ".UFICommentLink",
      ".UFIShareLink",
    ];

    commentSelectors.forEach((selector) => {
      const elements = clonedPost.querySelectorAll(selector);
      elements.forEach((el) => el.remove());
    });

    const mainContent = clonedPost.querySelector(
      ".xdj266r.x14z9mp.xat24cr.x1lziwak.x1vvkbs.x126k92a"
    );

    if (mainContent) {
      return extractTextFromHTML(mainContent.innerHTML);
    } else {
      const internalDivs = clonedPost.querySelectorAll("div[dir='auto']");
      let combinedText = "";

      internalDivs.forEach((div) => {
        const text = div.textContent.trim();
        if (text.length > 0) {
          combinedText += text + " ";
        }
      });

      if (combinedText.length > 0) {
        return combinedText.trim();
      }

      return extractTextFromHTML(clonedPost.innerHTML);
    }
  } catch (error) {
    console.warn("Erreur extraction contenu principal:", error);
    return extractTextFromHTML(postElement.innerHTML);
  }
}
\end{lstlisting}

\section{Fonctions de Validation}

\subsection{Vérification de Commentaire}

\begin{lstlisting}
function isComment(element) {
  const commentIndicators = [
    "UFIContainer",
    "UFIRow",
    "UFIComment",
    "comment",
    "commentContainer",
  ];

  const classList = element.className || "";
  if (commentIndicators.some((indicator) => classList.includes(indicator))) {
    return true;
  }

  const testId = element.getAttribute("data-testid") || "";
  if (testId.includes("comment") || testId.includes("ufi")) {
    return true;
  }

  const commentParent = element.closest(
    '.UFIContainer, [data-testid="ufi_comment_list"], .comments'
  );
  if (commentParent) {
    return true;
  }

  return false;
}
\end{lstlisting}

\subsection{Vérification de Sous-élément}

\begin{lstlisting}
function isSubElement(element) {
  const parentPost = element.closest(
    ".xdj266r.x14z9mp.xat24cr.x1lziwak.x1vvkbs.x126k92a"
  );
  if (parentPost && parentPost !== element) {
    return true;
  }

  return false;
}
\end{lstlisting}

\section{Fonctions de Gestion des IDs}

\subsection{Génération d'ID Simple}

\begin{lstlisting}
function generatePostId(index) {
  return `post-${index}`;
}
\end{lstlisting}

\subsection{Génération d'ID Unique}

\begin{lstlisting}
function generateUniquePostId(post) {
  const textContent = extractMainPostContent(post);
  const shortHash = btoa(
    encodeURIComponent(textContent.substring(0, 100))
  ).substring(0, 16);
  return `post-${shortHash}-${postIndex++}`;
}
\end{lstlisting}

\section{Fonctions de Gestion des Résultats}

\subsection{Vérification de Résultat Injecté}

\begin{lstlisting}
function hasInjectedResult(post) {
  return post.querySelector(".analysis-result") !== null;
}
\end{lstlisting}

\subsection{Nettoyage des Résultats}

\begin{lstlisting}
function clearInjectedResult(post) {
  const existingResult = post.querySelector(".analysis-result");
  if (existingResult) {
    existingResult.remove();
  }
}
\end{lstlisting}

\subsection{Injection de Résultat}

\begin{lstlisting}
function injectResult(post, postId, result) {
  clearInjectedResult(post);

  try {
    const resultLower = result.toLowerCase();

    let displayText = "";
    let backgroundColor = "";
    let borderColor = "";

    if (
      resultLower.indexOf("true") !== -1 &&
      resultLower.indexOf("expressive") === -1
    ) {
      displayText = "True";
      backgroundColor = "#008000";
      borderColor = "#008000";
    } else if (
      resultLower.indexOf("expressive true") !== -1 ||
      (resultLower.indexOf("expressive") !== -1 &&
        resultLower.indexOf("true") !== -1)
    ) {
      displayText = "Expressive True";
      backgroundColor = "#9ACD32";
      borderColor = "#9ACD32";
    } else if (
      resultLower.indexOf("expressive false") !== -1 ||
      (resultLower.indexOf("expressive") !== -1 &&
        resultLower.indexOf("false") !== -1)
    ) {
      displayText = "Expressive False";
      backgroundColor = "#f06900";
      borderColor = "#f06900";
    } else if (resultLower.indexOf("false") !== -1) {
      displayText = "False";
      backgroundColor = "#FF0000";
      borderColor = "#FF0000";
    } else {
      displayText = result;
      backgroundColor = "#666666";
      borderColor = "#666666";
    }

    const resultBox = document.createElement("div");
    resultBox.className = "analysis-result";
    resultBox.style.cssText = `
      margin-top: 10px;
      padding: 8px;
      background-color: ${backgroundColor};
      border: 1px solid ${borderColor};
      border-radius: 6px;
      font-size: 14px;
      color: white;
      font-weight: bold;
      z-index: 9999;
      position: relative;
    `;

    resultBox.textContent = `Analyse : ${displayText}`;

    const contentArea =
      post.querySelector(
        ".xdj266r.x14z9mp.xat24cr.x1lziwak.x1vvkbs.x126k92a"
      ) || post;

    contentArea.appendChild(resultBox);
    post.dataset.analyzed = "true";
    post.dataset.postId = postId;

    console.log(`Resultat injecte pour ${postId}: ${displayText}`);
  } catch (error) {
    console.error(`Erreur d'injection pour ${postId}:`, error);
  }
}
\end{lstlisting}

\section{Fonction d'Analyse}

\begin{lstlisting}
async function analyzePostWithId(text, postId) {
  const cacheKey = generateCacheKey(text);

  if (analysisCache.has(cacheKey)) {
    const cached = analysisCache.get(cacheKey);
    console.log(`Resultat trouve dans le cache pour ${postId}`);
    return cached.result;
  }

  try {
    console.log(`Analyse en cours pour ${postId}...`);

    try {
      const response = await fetch(
        `http://127.0.0.1:5000/ask?html=${encodeURIComponent(
          text
        )}&id=${encodeURIComponent(postId)}`,
        {
          method: "GET",
          mode: "cors",
          headers: {
            "Content-Type": "application/json",
          },
        }
      );

      if (!response.ok) throw new Error("Erreur de reponse du serveur");

      const data = await response.json();

      analysisCache.set(cacheKey, {
        result: data.response,
        timestamp: Date.now(),
        postId: postId,
      });

      saveCache();

      return data.response;
    } catch (error) {
      console.warn(
        "Serveur local non disponible, utilisation du mode de secours"
      );
      return "Post analyse (mode local non disponible)";
    }
  } catch (error) {
    console.error(`Erreur d'analyse pour ${postId} :`, error);
    return "Erreur d'analyse.";
  }
}
\end{lstlisting}

\section{Fonctions de Sélection des Posts}

\subsection{Sélecteurs de Posts}

\begin{lstlisting}
function getAllPostSelectors() {
  return [
    ".xdj266r.x14z9mp.xat24cr.x1lziwak.x1vvkbs.x126k92a",
  ];
}
\end{lstlisting}

\subsection{Récupération de Tous les Posts}

\begin{lstlisting}
function getAllPosts() {
  const allPosts = new Set();
  const selectors = getAllPostSelectors();

  selectors.forEach((selector) => {
    try {
      const posts = document.querySelectorAll(selector);
      posts.forEach((post) => {
        if (isComment(post) || isSubElement(post)) {
          return;
        }

        const textContent = extractMainPostContent(post);
        if (textContent && textContent.length > 10) {
          allPosts.add(post);
        }
      });
    } catch (error) {
      console.warn(`Erreur avec le selecteur ${selector}:`, error);
    }
  });

  console.log(`${allPosts.size} posts principaux trouves`);
  return Array.from(allPosts);
}
\end{lstlisting}

\section{Fonctions de Traitement}

\subsection{Observation des Changements}

\begin{lstlisting}
function observePostChanges() {
  const posts = getAllPosts();
  posts.forEach((post) => {
    const postId = post.dataset.postId;
  });
}
\end{lstlisting}

\subsection{Traitement Principal des Posts}

\begin{lstlisting}
async function processPosts() {
  const posts = getAllPosts();
  console.log(`Traitement de ${posts.length} posts...`);

  for (const post of posts) {
    let postId = post.id || post.dataset.postId;
    if (!postId) {
      postId = generateUniquePostId(post);
      post.id = postId;
    }

    if (pendingPosts.has(postId)) continue;

    const htmlContent = post.innerHTML;
    const cleanText = extractMainPostContent(post);

    if (!cleanText || cleanText.length < 10) continue;

    if (post.dataset.analyzed === "true" && hasInjectedResult(post)) {
      processedPosts.add(postId);
      continue;
    }

    const cacheKey = generateCacheKey(cleanText);
    if (analysisCache.has(cacheKey)) {
      const cached = analysisCache.get(cacheKey);
      console.log(`Injection depuis le cache pour ${postId}`);
      injectResult(post, postId, cached.result);
      processedPosts.add(postId);
      continue;
    }

    if (!processedPosts.has(postId)) {
      pendingPosts.add(postId);

      try {
        const result = await analyzePostWithId(cleanText, postId);
        injectResult(post, postId, result);
        processedPosts.add(postId);
      } catch (error) {
        console.error(`Erreur lors du traitement de ${postId}:`, error);
      } finally {
        pendingPosts.delete(postId);
      }
    }
  }
}
\end{lstlisting}

\section{Observateur de Scroll}

\begin{lstlisting}
function setupScrollObserver() {
  const observer = new MutationObserver((mutations) => {
    let hasNewContent = false;

    mutations.forEach((mutation) => {
      if (mutation.type === "childList" && mutation.addedNodes.length > 0) {
        mutation.addedNodes.forEach((node) => {
          if (node.nodeType === Node.ELEMENT_NODE) {
            const selectors = getAllPostSelectors();
            const isPost = selectors.some((selector) => {
              try {
                return node.matches(selector) || node.querySelector(selector);
              } catch (e) {
                return false;
              }
            });

            if (isPost) {
              hasNewContent = true;
            }
          }
        });
      }
    });

    if (hasNewContent) {
      setTimeout(processPosts, 500);
    }
  });

  observer.observe(document.body, {
    childList: true,
    subtree: true,
  });

  console.log("Observateur de scroll configure");
}
\end{lstlisting}

\section{Fonctions d'Initialisation}

\subsection{Traitement de Tous les Posts Visibles}

\begin{lstlisting}
async function processAllVisiblePosts() {
  await processPosts();
  console.log("Tous les posts visibles ont ete traites");
}
\end{lstlisting}

\subsection{Initialisation Principale}

\begin{lstlisting}
function initialize() {
  console.log("Initialisation de l'extension d'analyse de posts...");

  loadCache();
  setupScrollObserver();
  setTimeout(processAllVisiblePosts, 1000);
  setInterval(processAllVisiblePosts, 30000);

  console.log("Extension initialisee avec succes");
}
\end{lstlisting}

\section{Gestionnaire de Messages}

\begin{lstlisting}
chrome.runtime.onMessage.addListener((message, sender, sendResponse) => {
  console.log("Message recu:", message);

  if (message.action === "getStatus") {
    sendResponse({ active: true });
  } else if (message.action === "clearCache") {
    analysisCache.clear();
    localStorage.removeItem(CACHE_KEY);
    console.log("Cache vide");
    sendResponse({ success: true });
  } else if (message.action === "reprocessAll") {
    processedPosts.clear();
    const posts = getAllPosts();
    posts.forEach((post) => {
      post.dataset.analyzed = "false";
      clearInjectedResult(post);
    });
    processAllVisiblePosts();
    sendResponse({ success: true });
  }

  return true;
});
\end{lstlisting}

\section{Démarrage de l'Application}

\begin{lstlisting}
if (document.readyState === "loading") {
  document.addEventListener("DOMContentLoaded", initialize);
} else {
  initialize();
}
\end{lstlisting}

\end{document}
